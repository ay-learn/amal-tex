\begin{titlepage}
	\begin{center}
		\chapter{Les enjeux sociaux de l'intelligence artificielle}
		\minitoc

		\vspace{5cm}
		\pgfspectra[element=He,absorption]
	\end{center}
	\vfill % Remplir le reste de la page avec du blanc
\end{titlepage}
\pagestyle{monstyle}\setcounter{page}{7}

Les enjeux sociaux de l'intelligence artificielle et ses
applications sont innombrables. De la restructuration du marché du
travail : disparition d'emplois, évolution et création de nouveaux
emplois, à l'impact économique et quotidien de l'intelligence
artificielle.

\section{Les erreurs}
Les systèmes d'intelligence artificielle peuvent être défaillants et provoquer
une interruption de l'activité (pertes de données, erreurs, informations
incohérentes, etc.).
\section{Boit noirs}
Les approches numériques s'apparentent en revanche à une boîte
noire, incapable de justifier ses décisions : nul ne sait ce que
fait l'algorithme. Comment, dès lors, endosser la responsabilité
de la décision médicale ? Les données d'apprentissage sont en
particulier biaisées par les préjugés de l'époque et ceux des
concepteurs.
\section{Les problème philosophique}
Les tenants de l'intelligence artificielle dite forte visent à
concevoir une machine capable de raisonner comme l'humain, avec le
risque supposé de générer une machine supérieure à l'Homme et dotée
d'une conscience propre. Cette voie de recherche est toujours explorée
aujourd'hui, même si de nombreux chercheurs en \textbf{IA} estiment
qu'atteindre un tel objectif est impossible.

\section{Six principes pour garantir que l'IA fonctionne dans tous les pays pour le bien commun}

Afin d'atténuer les risques et de maximiser les opportunités
intrinsèques à l'utilisation de l'IA dans le domaine de la santé,
l'OMS propose que la réglementation et la gouvernance y afférentes
reposent sur les principes suivants :

\subsection{Protéger l'autonomie de l'être humain}
Dans le contexte des soins de santé, les individus doivent rester
maîtres des systèmes de soins de santé et des décisions médicales,
la vie privée et la confidentialité doivent être protégées et les
patients doivent donner un consentement éclairé valide au moyen de
cadres juridiques appropriés en matière de protection des données.

\subsection{Promouvoir le bien-être et la sécurité des personnes ainsi que l'intérêt public}
Les concepteurs de technologies d'IA doivent respecter les obligations
réglementaires relatives à la sécurité, à la précision et à l'efficacité pour
des utilisations ou des indications bien définies. Il faut pouvoir disposer de
mesures de contrôle de la qualité dans la pratique et d'amélioration de la
qualité dans l'utilisation de l'IA.

\subsection{Garantir la transparence, la clarté et l'intelligibilité}
La transparence exige que des informations suffisantes soient publiées ou
documentées avant la conception ou le déploiement d'une technologie d'IA. Ces
informations doivent être facilement accessibles et permettre une consultation
et un débat publics constructifs sur la conception de la technologie et sur
l'utilisation qui doit ou non en être faite.

\subsection{Encourager la responsabilité et l'obligation de rendre des comptes}
Même si les technologies d'IA permettent d'accomplir des tâches spécifiques, il
incombe aux parties prenantes de veiller à ce qu'elles soient utilisées dans
des conditions appropriées et par des personnes dûment formées. Des mécanismes
efficaces doivent être mis en place pour permettre aux individus et aux groupes
lésés par des décisions fondées sur des algorithmes de contester ces décisions
et d'obtenir réparation.

\subsection{Garantir l'inclusion et l'équité}
L'inclusion suppose que l'IA appliquée à la santé soit conçue de manière à
encourager l'utilisation et l'accès équitables les plus larges possibles,
indépendamment de l'âge, du sexe, du genre, des revenus, de la race, de
l'origine ethnique, de l'orientation sexuelle, des capacités ou d'autres
caractéristiques protégées par les codes relatifs aux droits humains.

\subsection{Promouvoir une IA réactive et durable}
Les concepteurs, les développeurs et les utilisateurs devraient évaluer de
manière continue et transparente les applications de l'IA en situation réelle
afin de s'assurer que cette technologie répond de manière adéquate et
appropriée aux attentes et aux besoins. Les systèmes d'IA devraient également
être conçus de sorte à réduire au minimum leurs conséquences environnementales
et à accroître leur efficacité énergétique. Les gouvernements et les
entreprises devraient anticiper les bouleversements qui seront occasionnés au
niveau du travail, notamment la formation des agents de santé qui devront se
familiariser avec l'utilisation des systèmes d'IA, et les pertes d'emploi que
le recours à des systèmes automatisés est susceptible d'engendrer.

\bigskip
Ces principes guideront les travaux futurs de l'OMS en vue de garantir que le
plein potentiel de l'IA en matière de soins de santé et de santé publique sera
mis au service du bien de tous.

